\documentclass[fleqn, 10pt]{article}
\usepackage[utf8]{inputenc}
\usepackage[russian, english]{babel}
\usepackage[T2A]{fontenc}
\usepackage{mathtools}
\usepackage{units}
\usepackage{pgfplots}
\pgfplotsset{compat=1.9}
\usepgfplotslibrary{fillbetween}
\newcommand\abs[1]{\left|#1\right|}

\title{Типовой расчет по Теории Вероятности}
\author{Зенкова Дарья, M3336}
\date{Вариант 2}



\begin{document}
\pagenumbering{gobble}
\maketitle
\newpage
\pagenumbering{arabic}

\section{Непосредственный подсчёт вероятностей в рамках классической схемы. Теоремы сложения и умножения.}

\paragraph{Задача}
В коробке лежат карандаши: двенадцать красных и восемь зеленых. Наудачу извлекают три. Какова вероятность того, что среди извлечённых будет хотя бы один красный карандаш?
\subparagraph{Решение}
\[N_{\text{кр}} =\ 12\] 
\[N_{\text{з}} =\ 8\]
\[\Omega =\ \{\text{Все возможные сочетания трех карандашей}\}\]
\[\abs{\Omega} =\ C^3_{20} = \frac{20!}{3!\cdot17!} = \frac{18\cdot19\cdot20}{6} = 3\cdot19\cdot20\ = 1140 \]
\[A =\ \{\text{Хотя бы один из трех карандашей - красный}\} \]
\[\overline{A} =\ \{\text{Среди трех карандашей нет ни одного красного}\} \]
\[\abs{\overline{A}} =\ C^3_{8} = \frac{8!}{3!\cdot5!} = \frac{6\cdot7\cdot8}{6} = 56\]
\[p(A) =\ 1 - p(\overline{A}) = 1 - \frac{\abs{\overline{A}}}{\abs{\Omega}} = 1 - \frac{56}{1140} = \frac{271}{285}\]
\subparagraph{Ответ} \[\frac{271}{285}\]
\newpage

\section{Геометрические вероятности.}
\paragraph{Задача}
Из промежутка \({[{-2}, 2]}\) наудачу выбраны два числа \(\xi_1\) и \(\xi_2\). Найти вероятность того, что квадратное уравнение \(x^2+\xi_1x+\xi_2=0\) будет иметь вещественные корни.
\subparagraph{Решение}
\[\Omega =\ \{{-2}\leq\xi_1\leq2, {-2}\leq\xi_2\leq2\}\]
Уравнение \(x^2+\xi_1x+\xi_2=0\) имеет вещественные корни \(\iff D = \xi_1^2 - 4\xi_2 \geq 0\). \\
Таким образом, \(A = \ \{\xi_2 \leq \frac{\xi_1^2}{4}, {-2}\leq\xi_1\leq2, {-2}\leq\xi_2\leq2\}\).\\

\begin{tikzpicture} 
  \begin{axis}[
      axis equal, axis lines = middle, axis line style ={->},
	xlabel = {$\xi_1$},
	ylabel = {$\xi_2$},
  xmin = -2.5,
  xmax = 2.5,
  ymin = -2.5,
  ymax = 2.5]
    \addplot[name path = square,domain = -4.0:4.0,color=blue,mark=none] coordinates {
      (-2,-2)
      (-2, 2)
      (2, 2)
      (2, -2)
      (-2,-2)
      } \closedcycle;
    \addplot[name path = f,domain=-4.0:4.0,color=red,mark=none]{x^2 / 4};
    \addplot[blue, fill opacity = 0.1] fill between[of = square and f, soft clip = {domain=-2:2},];
    
    
\end{axis}
\end{tikzpicture}
\[\mu(\Omega) = (2 - (-2))\cdot(2 - (-2)) = 16\]
\[\mu(A) = (2 - (-2))\cdot 2 + \int_{-2}^{2}\frac{\xi_1^2}{4}\,\mathrm{d}\xi_1 = 8 + \left.\frac{\xi_1^3}{12}\right|_{-2}^2 = 8 + \frac{8}{12} - (-\frac{8}{12}) = \frac{28}{3}\]
\[p(A) = \frac{\mu(A)}{\mu(\Omega)} = \frac{28}{3\cdot16} = \frac{7}{12}\]
\subparagraph{Ответ}
\[\frac{7}{12}\]
\section{Формула полной вероятности. Формула Байеса.}
\paragraph{Задача}
Два стрелка \(A\) и \(B\) поочередно стреляют в мишень до первого попадания, но не более двух раз каждый. Вероятноть попадания при одном выстреле для \(A\) равна \(0.8\), для \(B\) -- \(0.6\). Первый стрелок опеределяется по жребию. Для этого кидается игральный кубик. Если выпадает число, кратное трём, то начинает \(A\), иначе первым стреляет \(B\). В результате стрельбы выиграл стрелок \(B\). Какова вероятность, что он стрелял первым?
\paragraph{Решение}
  
\section{Схема Бернулли.}
\paragraph{Задача}
Производится четыре выстрела по мишени, вероятность попадания при каждом выстреле равна \(\nicefrac{2}{3}\). Найти вероятность того, что в мишень попадут не менее двух раз.
\paragraph{Решение}

\end{document}
