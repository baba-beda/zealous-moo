\documentclass[fleqn]{article}
\usepackage[utf8]{inputenc}
\usepackage[russian, english]{babel}
\usepackage[T2A]{fontenc}
\usepackage{mathtools}
\usepackage{units}
\usepackage{pgfplots}

\title{Типовой расчет по Теории Вероятности}
\author{Зенкова Дарья, M3336}
\date{Вариант 2}



\begin{document}
\pagenumbering{gobble}
\maketitle
\newpage
\pagenumbering{arabic}

\section{Непосредственный подсчёт вероятностей в рамках классической схемы. Теоремы сложения и умножения.}

\paragraph{Задача}
В коробке лежат карандаши: двенадцать красных и восемь зеленых. Наудачу извлекают три. Какова вероятность того, что среди извлечённых будет хотя бы один красный карандаш?
\subparagraph{Решение}
\begin{align*}
N_{\text{кр}} =&\ 12 \\ 
N_{\text{з}} =&\ 8 \\
\Omega =&\ \{\text{Все возможные сочетания трех карандашей}\} \\
\textbf{card}\ \Omega =&\ C^3_{20} = \frac{20!}{3!\cdot17!} = \frac{18\cdot19\cdot20}{6} = 3\cdot19\cdot20\ = 1140 \\
A =&\ \{\text{Хотя бы один из трех карандашей - красный}\} \\
\overline{A} =&\ \{\text{Среди трех карандашей нет ни одного красного}\} \\
\textbf{card}\ \overline{A} =&\ C^3_{8} = \frac{8!}{3!\cdot5!} = \frac{6\cdot7\cdot8}{6} = 56 \\
p(A) =&\ 1 - p(\overline{A}) = 1 - \frac{\textbf{card}\ \overline{A}}{\textbf{card}\ \Omega} = 1 - \frac{56}{1140} = \frac{271}{285}
\end{align*}
\subparagraph{Ответ} \(\frac{271}{285}\)
  
\section{Геометрические вероятности.}
\paragraph{Задача}
Из промежутка \({[{-2}, 2]}\) наудачу выбраны два числа \(\xi_1\) и \(\xi_2\). Найти вероятность того, что квадратное уравнение \(x^2+\xi_1x+\xi_2=0\) будет иметь вещественные корни.
\paragraph{Решение}
  
\section{Формула полной вероятности. Формула Байеса.}
\paragraph{Задача}
Два стрелка \(A\) и \(B\) поочередно стреляют в мишень до первого попадания, но не более двух раз каждый. Вероятноть попадания при одном выстреле для \(A\) равна \(0.8\), для \(B\) -- \(0.6\). Первый стрелок опеределяется по жребию. Для этого кидается игральный кубик. Если выпадает число, кратное трём, то начинает \(A\), иначе первым стреляет \(B\). В результате стрельбы выиграл стрелок \(B\). Какова вероятность, что он стрелял первым?
\paragraph{Решение}
  
\section{Схема Бернулли.}
\paragraph{Задача}
Производится четыре выстрела по мишени, вероятность попадания при каждом выстреле равна \(\nicefrac{2}{3}\). Найти вероятность того, что в мишень попадут не менее двух раз.
\paragraph{Решение}

\end{document}
