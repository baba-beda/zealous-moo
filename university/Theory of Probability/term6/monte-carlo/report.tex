\documentclass[fleqn, 10pt]{article}
\usepackage[utf8]{inputenc}
\usepackage[russian, english]{babel}
\usepackage[T2A]{fontenc}
\usepackage{mathtools}
\usepackage{units}
\newcommand\abs[1]{\left|#1\right|}
\usepackage{pgfplots}
\usepackage{multirow}
\usetikzlibrary{intersections}

\title{Метод Монте-Карло}
\author{Зенкова Дарья, M3336}
\date{Вариант 2}

\begin{document}
\pagenumbering{gobble}
\maketitle
\pagenumbering{arabic}

\section{Оценка объема}
\subsection{Задача}
Методом Монте-Карло оценить объем части тела \(\{F(\overline{x}) \leq c\}\), заключенной в \(k\)-мерном кубе с ребром \([0, 1]\). Функция имеет вид \(F(\overline{x}) = f(x_1) + f(x_2) + ... + f(x_k)\). Для выбранной надежности \(\gamma \geq 0.95\) указать асимптотическую точность оценивания и построить асимптотический доверительный интервал для истинного значения объема.\\
\(f(x) = x^3\), размерность \(k = 6\), \(c = 1.4\), объем выборки \(n = 10^4\) и \(n = 10^6\).
\subsection{Решение}
Порядок действий:
\begin{itemize}
\item[1)] Для начала сгенерируем \(n\) случайных точек в заданном \(k\)-мерном кубе.
\item[2)] Вычислим функцию \(F(\overline{x})\) в каждой из точек.
\item[3)] Заведем множество \(In = \{p_i | 0 \ if \ F(\overline{x_i}) > c,1\ otherwise\}\).
\item[4)] Вычислим среднее: \(mean(In) = m\). Это и будет математическим ожиданием объема тела.
\item[5)] Далее найдем границы доверительного интервала. Для этого нам понадобится \(\sigma\) - среднеквадратичное отклонение, \(t : \Phi(t) = \frac{\gamma}{2}\). Для данного \(\gamma\) \(t = 0.64\). Доверительный интервал выглядит так: \[m - t\frac{\sigma}{n} \leq V \leq m + t\frac{\sigma}{n}\].
  
\end{itemize}
\subsection{Результат}
\[n = 10^4: \ 0.470005 \leq V \leq 0.476395\]
\[n = 10^6: \ 0.473600 \leq V \leq 0.474240\]

\section{Оценка интеграла}
\subsection{Задача}
Построить оценку интегралов:
\[\text{a)} \int\limits_{2}^{5}{ln(1+x^2)dx}\]
\[\text{b)} \int\limits_{-\infty}^{\infty}{cos(x)e^{-\frac{(x+3)^2}{4}}dx}\]
\subsection{Решение}
Приближение интеграла можно рассматривать так:
\[\int\limits_{a}^{b}{f(x)dx} = (b - a)\textbf{E}(f(x))\text{, }\]
где математическое ожидание функции рассматривается как среднее по полученным значениям на выборке.\\
Второй интеграл имеет бесконечные границы, соответственно необходимо сделать такую замену, что границы будут конечными.
\[\int\limits_{-\infty}^{\infty}{cos(x)e^{-\frac{(x+3)^2}{4}}dx} = \int\limits_{-\infty}^{-3}{cos(x)e^{-\frac{(x+3)^2}{4}}dx} + \int\limits_{-3}^{\infty}{cos(x)e^{-\frac{(x+3)^2}{4}}dx}\]
Сделаем в первом интеграле замену \(u = \frac{x+3}{2}\):
\[\int\limits_{-\infty}^{-3}{cos(x)e^{-\frac{(x+3)^2}{4}}dx} = 2\int\limits_{-\infty}^0{cos(2u-3)e^{-u^2}du}\]
Далее сделаем замену \(t = -u\):
\[\int\limits_{-\infty}^0{cos(2u-3)e^{-u^2}} = \int\limits_{0}^{\infty}{cos(-2t-3)e^{-t^2}}dt\]
Замена \(r = \frac{1}{t+1}\).
\[\int\limits_{0}^{\infty}{cos(-2t-3)e^{-t^2}dt} = \int\limits_{0}^{1}{cos(-\frac{2}{r}-1)e^{-\frac{(1-r)^2}{r^2}}\frac{1}{r^2}dr}\]
Аналогично:
\[\int\limits_{-3}^{\infty}{cos(x)e^{-\frac{(x+3)^2}{4}}dx} = 2\int\limits_{0}^{1}{cos(\frac{2}{r}-5)e^{-\frac{(1-r)^2}{r^2}}\frac{1}{r^2}dr}\]
В итоге:
\[\int\limits_{-\infty}^{\infty}{cos(x)e^{-\frac{(x+3)^2}{4}}dx} = 2\int\limits_{0}^{1}{(cos(\frac{2}{r}-5) + cos(-\frac{2}{r}-1))e^{-\frac{(1-r)^2}{r^2}}\frac{1}{r^2}dr}\]
\subsection{Результат}
\begin{itemize}
\item[a)] \(I = \int\limits_{2}^{5}{ln(1+x^2)dx} \approx 7.6041\)
\[n=10^4: \ 7.586897 \leq I \leq 7.605158\]
\[n=10^6: \ 7.601046 \leq I \leq 7.602853\]
\item[b)] \(I = \int\limits_{-\infty}^{\infty}{cos(x)e^{-\frac{(x+3)^2}{4}}dx}\approx -1.29105\)
\[n=10^4: \ -1.319270 \leq I \leq -1.284359 \]
\[n=10^6: \ -1.293289 \leq I \leq -1.289793\]
\end{itemize}

\end{document}
