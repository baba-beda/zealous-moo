\documentclass[fleqn, 10pt]{article}
\usepackage[utf8]{inputenc}
\usepackage[russian, english]{babel}
\usepackage[T2A]{fontenc}
\usepackage{mathtools}
\usepackage{units}
\newcommand\abs[1]{\left|#1\right|}
\usepackage{pgfplots}
\usepackage{multirow}
\usetikzlibrary{intersections}

\title{Типовой расчет по Теории Вероятностей}
\author{Зенкова Дарья, M3336}
\date{Вариант 2}

\begin{document}
\pagenumbering{gobble}
\maketitle
\newpage
\pagenumbering{arabic}

\section{Функции случайных величин}
\subsection{Задача}
Функция распределения \(F_X(t)\) случайной величины \(X\) имеет вид:
\[F_X(t) =
\begin{cases}
  1 - e^{-t}, & t \geq 0 \\
  0, & t < 0
\end{cases}\]
Случайные величины \(Y = X^2\) и \(Z=-3X+2\) являются функциями от случайной величины \(X\).
Найти:
\begin{description}
\item[а)] плотность распределения \(f_y(v)\) случайной величины \(Y\);
\item[б)] моменты \(\textbf{E}Z\), \(\textbf{D}Z\), \(\textbf{cov}(X,Z)\).
\end{description}
\subsection{Решение}
\begin{itemize}
\item[а)] Пусть \(D_v\) - множество значений случайной величины \(X\), для которых \(y=g(x)=x^2 < v\): \(D_v=\{x : y = x^2 < v\}\). \\
При \(v \leq 0\) \(D_v = \emptyset \implies\) \mbox{\(F_Y(v) =\textbf{P}\{Y\in D_v\} = 0\)} \(\implies f_Y(v) = 0\) при \(v \leq 0\).77
При \(v > 0\) \(D_v\) имеет вид:
\[D_v = \{x : x^2 < v\} = \{x : -\sqrt{v} < x < \sqrt{v}\}\]
Тогда функция распределения \(Y\):
\[F_Y(v) = \textbf{P}\{-\sqrt{v} < X < \sqrt{v}\} = F_X(\sqrt{v}) - F_X(-\sqrt{v}) = 1 - e^{\sqrt{v}}\]
Дифференцируя по \(v\), получим плотность распределения \(Y\):
\[f_Y(v) = (1 - e^{\sqrt{v}})' = -e^{\sqrt{v}}\cdot\frac{1}{2\sqrt{v}}\]
Таким образом,
\[f_Y(v) =
\begin{cases}
  -\frac{e^{\sqrt{v}}}{2\sqrt{v}}, & v > 0 \\
  0, & v \leq 0
\end{cases}
\]
\item[б)] Пусть \(Z = -3X + 2\).
  Для начала найдем плотность распределения \(X\):
  \[f_X(t) =
  \begin{cases}
    e^{-t}, & t \geq 0 \\
    0, & t < 0
  \end{cases}
  \]
  Математическое ожидание случайной величины \(X\):
  \[\textbf{E}X = \int\limits_{-\infty}^{\infty}{tf_X(t)dt} = \int\limits_{0}^{\infty}{te^{-t}dt} = 1\]
  Математическое ожидание \(X^2\):
  \[\textbf{E}(X^2) = \int\limits_{0}^{\infty}{t^2e^{-t}dt} = 2\]
  Дисперсия случайной величины \(X\):
  \[\textbf{D}X = \textbf{E}(X^2) - (\textbf{E}X)^2 = 2 - 1 = 1\]
  Воспользуемся линейностью математического ожидания, чтобы найти \(\textbf{E}Z\):
  \[\textbf{E}Z = \textbf{E}(-3X+2) = -3\textbf{E}X + 2 = -1\].
  Дисперсия \(Z\):
  \[\textbf{D}Z = \textbf{D}(-3X+2) = 9\textbf{D}X = 9\]
  Воспользуемся следующим свойством ковариации:
  \[\textbf{cov}(aX+b, cY+d) = ab\textbf{cov}(X,Y)\]
  Тогда ковариация случайных величин \(X\) и \(Z\):
  \[\textbf{cov}(X, Z) = \textbf{cov}(X, -3X+2) = -3\textbf{cov}(X, X) = -3\textbf{D}X = -3\]
  \end{itemize}
\subsection{Ответ}
\begin{itemize}
\item[а)]
  \[f_Y(v) =
\begin{cases}
  -\frac{e^{\sqrt{v}}}{2\sqrt{v}}, & v > 0 \\
  0, & v \leq 0
\end{cases}
\]
\item[б)] \(\textbf{E}Z = -1\), \(\textbf{D}Z = 9\), \(\textbf{cov}(X,Z) = -3\)
\end{itemize}

\section{Регрессия. Случай дискретных случайных величин}
\subsection{Задача}
Случайный вектор \((X,Y)\) задан матрией распределения:
\begin{center}
\begin{tabular}{
    | c | c | c | c | c | }
  \hline
  \multirow{2}{1em}{\(Y\)} & \multicolumn{4}{|c|}{\(X\)} \\
  \cline{2-5}
  & \(-2\) & \(0\) & \(2\) & \(4\) \\
  \hline
  \(-2\) & \(0.05\) & \(0.05\) & \(0.2\) & \(0.1\) \\
  \hline
  \(2\) & \(0.25\) & \(0.15\) & \(0.1\) & \(0.1\) \\
  \hline
\end{tabular}
\end{center}
Найти условные ожидания \(\textbf{E}(X|Y)\) и \(\textbf{E}(Y|X)\), проверить формулу полного математического ожидания. Построить линейную регрессию \(X\) на \(Y\) и \(Y\) на \(X\) и вычислить значения этих функций в точках \(x_i\) и \(y_j\). Геометрически сравнить значения регрессии и линейной регрессии.
\subsection{Решение}
\[\textbf{E}(Y|X) = \textbf{E}(Y|T=\{X^{-1}(x_j)\}) = \sum\limits_j^{}\textbf{E}(Y|X^{-1}(x_j))I_{X^{-1}(x_j)} = \sum\limits_j^{}\textbf{E}(Y|X=x_j)I_{C_j}\text{,}\]
где \(T = \{C_j\}_j \text{, } C_j = X^{-1}(x_j) = \{\omega : X(\omega) = x_j\} \text{, } x_i \ne x_j \ \forall i \ne j\).
Найдем условное математическое ожидание \(\textbf{E}(Y|X) = m_{Y|X}(x)\):
\[\textbf{E}(Y|X=x_i) = y_1\cdot\frac{\textbf{P}(Y=y_1|X=x_i)}{\textbf{P}(X=x_i)} + y_2 \cdot\frac{\textbf{P}(Y=y_2|X=x_i)}{\textbf{P}(X=x_i)}\]
\[\textbf{E}(Y|X=-2) = -2\cdot\frac{0.05}{0.05+0.25} + 2\cdot\frac{0.25}{0.05+0.25} = \frac{4}{3}\]
\[\textbf{E}(Y|X=0) = 1\]
\[\textbf{E}(Y|X=2) = -\frac{2}{3}\]
\[\textbf{E}(Y|X=4) = 0\]
\[\textbf{E}(Y|X) = \frac{4}{3}I_{X=-2}+I_{X=0}-\frac{2}{3}I_{X=2}\]
Найдем условное математическое ожидание \(\textbf{E}(X|Y)=m_{X|Y}(y)\):
\[\textbf{E}(X|Y=-2) = \frac{7}{4}\]
\[\textbf{E}(X|Y=2) = \frac{1}{6}\]
\[\textbf{E}(X|Y) = \frac{7}{4}I_{Y=-2}+\frac{1}{6}I_{Y=2}\]
\paragraph{Def} \(g^{*}(x)\) - функция линейной регрессии \(Y\) на \(X\), если:
\[\inf\limits_{g(x)=ax+b}{\textbf{E}(Y-g(X))^2} = \textbf{E}(Y-g^{*}(X))^2\]
\[g^{*}(x) = a^{*}x+b^{*}\]
\[a^{*} = \frac{\textbf{cov}(X,Y)}{\textbf{D}X}\text{, } b^{*} = \textbf{E}Y - a^{*}\textbf{E}X\]
Пусть \(g(x) = ax + b\) - линейная регрессия \(Y\) на \(X\), \(h(y) = cy + d\) - линейная регрессия \(X\) на \(Y\). Для начала найдем все необходимые моменты:
\[\textbf{cov}(X,Y) = \textbf{E}((X-\textbf{E}X)(Y-\textbf{E}Y)) = \textbf{E}(XY) - \textbf{E}X\cdot\textbf{E}Y\]
\[\textbf{E}(XY) = \sum\limits_{i=1}^{4}{\sum\limits_{j=1}^{2}{x_iy_j\textbf{P}(X=x_i,Y=y_j)}} = -1.2\]
\[\textbf{E}X = \sum\limits_{i=1}^4{x_i\textbf{P}(X=x_i)} = 0.8\]
\[\textbf{E}Y = \sum\limits_{j=1}^2{y_j\textbf{P}(Y=y_j)} = 0.4\]
\[\textbf{cov}(X,Y) = -1.2 - 0.8\cdot0.4 = -1.52\]
\[\textbf{E}(X^2) = 5.6\]
\[\textbf{E}(Y^2) = 4\]
\[\textbf{D}X = \textbf{E}(X^2) - (\textbf{E}X)^2 = 5.6 - 0.64 = 4.96\]
\[\textbf{D}Y = \textbf{E}(Y^2) - (\textbf{E}Y)^2 = 4 - 0.16 = 3.84\]
Теперь можем найти все коэффициенты:
\[a = \frac{\textbf{cov}(X, Y)}{\textbf{D}X} = \frac{-1.52}{4.96} = -\frac{19}{62}\]
\[b = \textbf{E}Y - a\textbf{E}X = 0.4 + \frac{19}{62}\cdot0.8 = \frac{20}{31}\]
\[c = \frac{\textbf{cov}(X, Y)}{\textbf{D}Y} = \frac{-1.52}{3.84} = -\frac{19}{48}\]
\[d = \textbf{E}X - c\textbf{E}Y = 0.8  + \frac{19}{48}\cdot0.4 = \frac{23}{24}\]
Таким образом, искомые линейные регрессии имеют следующий вид:
\[g(x) = -\frac{19}{62}x+\frac{20}{31}\]
\[h(y) = -\frac{19}{48}y + \frac{23}{24}\]
Сравним получившиеся регрессии и линейные регрессии.

\begin{center}
\begin{tikzpicture}
  \begin{axis}[
      title=\(Y\) на \(X\),
      axis lines = middle, axis line style = {->},
      xlabel={x},
      ylabel={t},
      xtick={-2,-1,0,1,2,3,4},
      ytick={-4,-3,-2,-1,0,1,2,3,4}]
    \addplot[color=blue,domain=-2.1:4.1,mark=none]{-19/62*x + 20/31};
    \legend{Линейная регрессия, Регрессия};
    \addplot[color=red,mark=*] coordinates {
      (-2, 4/3)
      (0, 1)
      (2, -2/3)
      (4,0)
    };
  \end{axis}
\end{tikzpicture}
\end{center}

\begin{center}
\begin{tikzpicture}
  \begin{axis}[
      title=\(X\) на \(Y\),
      axis lines = middle, axis line style = {->},
      xlabel={y},
      xtick={-2,-1,0,1,2},
      ytick={-2,-1,0,1,2,3,4,5,6},
      ylabel={t}]
    \addplot[color=blue,domain=-2.1:2.1,mark=none]{-19/48*x+23/24};
    \addplot[color=red,mark=*] coordinates {
      (-2, 7/4)
      (2, 1/6)
    };
    \legend{Линейная регрессия, Регрессия};
  \end{axis}
\end{tikzpicture}
\end{center}
\subsection{Ответ}
\[E(Y|X) = m_{Y|X}(x) =  \frac{4}{3}I_{X=-2}+I_{X=0}-\frac{2}{3}I_{X=2}\]
\[E(X|Y) = m_{X|Y}(y) =  \frac{7}{4}I_{Y=-2}+\frac{1}{6}I_{Y=2}\]
\[g(x) = -\frac{180}{119}x+\frac{958}{595}\]
\[h(y) = -\frac{90}{49}y + \frac{54}{35}\]

\end{document}
