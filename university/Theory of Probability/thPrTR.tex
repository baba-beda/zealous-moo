\documentclass[fleqn, 10pt]{article}
\usepackage[utf8]{inputenc}
\usepackage[russian, english]{babel}
\usepackage[T2A]{fontenc}
\usepackage{mathtools}
\usepackage{units}
\newcommand\abs[1]{\left|#1\right|}
\usepackage{pgfplots}
\usepackage{multirow}

\title{Типовой расчет по Теории Вероятностей}
\author{Зенкова Дарья, M3336}
\date{Вариант 2}

\begin{document}
\pagenumbering{gobble}
\maketitle
\newpage
\pagenumbering{arabic}

\section{Функции случайных величин}
\subsection{Задача}
Функция распределения \(F_X(t)\) случайной величины \(X\) имеет вид:
\[F_X(t) =
\begin{cases}
  1 - e^{-t}, & t \geq 0 \\
  0, & t < 0
\end{cases}\]
Случайные величины \(Y = X^2\) и \(Z=-3X+2\) являются функциями от случайной величины \(X\).
Найти:
\begin{description}
\item[а)] плотность распределения \(f_y(v)\) случайной величины \(Y\);
\item[б)] моменты \(\textbf{E}Z\), \(\textbf{D}Z\), \(\textbf{cov}(X,Z)\).
\end{description}
\subsection{Решение}
\begin{itemize}
\item[а)] Пусть \(D_v\) - множество значений случайной величины \(X\), для которых \(y=g(x)=x^2 < v\): \(D_v=\{x : y = x^2 < v\}\). \\
При \(v \leq 0\) \(D_v = \emptyset \implies\) \mbox{\(F_Y(v) =\textbf{P}\{Y\in D_v\} = 0\)} \(\implies f_Y(v) = 0\) при \(v \leq 0\).
При \(v > 0\) \(D_v\) имеет вид:
\[D_v = \{x : x^2 < v\} = \{x : -\sqrt{v} < x < \sqrt{v}\}\]
Тогда функция распределения \(Y\):
\[F_Y(v) = \textbf{P}\{-\sqrt{v} < X < \sqrt{v}\} = F_X(\sqrt{v}) - F_X(-\sqrt{v}) = 1 - e^{\sqrt{v}}\]
Дифференцируя по \(v\), получим плотность распределения \(Y\):
\[f_Y(v) = (1 - e^{\sqrt{v}})' = -e^{\sqrt{v}}\cdot\frac{1}{2\sqrt{v}}\]
Таким образом,
\[f_Y(v) =
\begin{cases}
  -\frac{e^{\sqrt{v}}}{2\sqrt{v}}, & v > 0 \\
  0, & v \leq 0
\end{cases}
\]
\item[б)] Пусть \(Z = -3X + 2\).
  Для начала найдем плотность распределения \(X\):
  \[f_X(t) =
  \begin{cases}
    e^{-t}, & t \geq 0 \\
    0, & t < 0
  \end{cases}
  \]
  Математическое ожидание случайной величины \(X\):
  \[\textbf{E}X = \int\limits_{-\infty}^{\infty}{tf_X(t)dt} = \int\limits_{0}^{\infty}{te^{-t}dt} = 1\]
  Математическое ожидание \(X^2\):
  \[\textbf{E}(X^2) = \int\limits_{0}^{\infty}{t^2e^{-t}dt} = 2\]
  Дисперсия случайной величины \(X\):
  \[\textbf{D}X = \textbf{E}(X^2) - (\textbf{E}X)^2 = 2 - 1 = 1\]
  Воспользуемся линейностью математического ожидания, чтобы найти \(\textbf{E}Z\):
  \[\textbf{E}Z = \textbf{E}(-3X+2) = -3\textbf{E}X + 2 = -1\].
  Дисперсия \(Z\):
  \[\textbf{D}Z = \textbf{D}(-3X+2) = 9\textbf{D}X = 9\]
  Воспользуемся следующим свойством ковариации:
  \[\textbf{cov}(aX+b, cY+d) = ab\textbf{cov}(X,Y)\]
  Тогда ковариация случайных величин \(X\) и \(Z\):
  \[\textbf{cov}(X, Z) = \textbf{cov}(X, -3X+2) = -3\textbf{cov}(X, X) = -3\textbf{D}X = -3\]
  \end{itemize}
\subsection{Ответ}
\begin{itemize}
\item[а)]
  \[f_Y(v) =
\begin{cases}
  -\frac{e^{\sqrt{v}}}{2\sqrt{v}}, & v > 0 \\
  0, & v \leq 0
\end{cases}
\]
\item[б)] \(\textbf{E}Z = -1\), \(\textbf{D}Z = 9\), \(\textbf{cov}(X,Z) = -3\)
\end{itemize}

\section{Регрессия. Случай дискретных случайных величин}
\subsection{Задача}
Случайный вектор \((X,Y)\) задан матрией распределения:
\begin{center}
\begin{tabular}{
    | c | c | c | c | c | }
  \hline
  \multirow{2}{1em}{\(Y\)} & \multicolumn{4}{|c|}{\(X\)} \\
  \cline{2-5}
  & \(-2\) & \(0\) & \(2\) & \(4\) \\
  \hline
  \(-2\) & \(0.05\) & \(0.05\) & \(0.2\) & \(0.1\) \\
  \hline
  \(2\) & \(0.25\) & \(0.15\) & \(0.1\) & \(0.1\) \\
  \hline
\end{tabular}
\end{center}
Найти условные ожидания \(\textbf(E)(X|Y)\) и \(\textbf{E}(Y|X)\), проверить формулу полного математического ожидания. Построить линейную регрессию \(X\) на \(Y\) и \(Y\) на \(X\) и вычислить значения этих функций в точках \(x_i\) и \(y_j\). Геометрически сравнить значения регрессии и линейной регрессии.
\subsection{Решение}

\end{document}
